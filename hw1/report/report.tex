\documentclass[a4paper, 10pt]{article}

\usepackage{algpseudocode}
\usepackage{amsmath}
\usepackage{fontspec}
\usepackage{hyperref}

\setmainfont{Open Sans}
\setmonofont{Hack}

\title{Homework 1 - Sokoban}
\author{106062202}
\date{\today}

\begin{document}
    \maketitle

    \section{Problem Description}
        In the game Sokoban, the player is able to push boxes in the direction of movement.
        The goal is to push all boxes onto target tiles. 
        \href{http://lms.nthu.edu.tw/course.php?courseID=43477&f=hw&hw=214237}{(Homework link)}.
        \paragraph{Input}
            The map of a game in a text file, where the each character represents a tile.
            \begin{itemize}
                \item \texttt{ } (space): A regular tile.
                \item \texttt{\#}: A wall block.
                \item \texttt{.}: A target tile onto which the box is to be pushed.
                \item \texttt{o}: The player. Capitalized (\texttt{O}) when it is stepping on a target tile.
                \item \texttt{x}: A box. Capitalized (\texttt{X}) when it is pushed onto a target tile.
            \end{itemize}
        \paragraph{Output}
            A sequence of actions, represented in \texttt{WASD}.

    \section{Implementation}

        \subsection{Overall Algorithm}
            \begin{algorithmic}[1] 
                \State $stateQueue \gets (H(initState), initState)$ \label{stateQueue}
                \State $visited \gets initMap$
                \While {$stateQueue$ is not empty}
                    \State $currState \gets stateQueue.pop()$
                    \State $substateList \gets findSubstates(currState)$ \label{findSubstates}
                    \For {$substate \in substateList$}
                        \State $newState \gets pushBox(currState, substate)$ \label{Update}
                        \If {$newState$ is solved}
                            \State return $currState.seq + substate.seq$ \label{seq}
                        \EndIf
                        \If {$newState \notin visited$}
                            \State $visited \gets newMap$
                            \If {pushed box is not \texttt{'X'} \textbf{AND} $newState$ is not deadlock} \label{deadlock}
                                \State $stateQueue \gets (H(newState), newState)$
                            \EndIf
                        \EndIf
                    \EndFor
                \EndWhile
            \end{algorithmic}

            \begin{itemize}
                \item In line \ref{stateQueue}, a priority queue is used to 
                    prioritize states that are heuristically closer to solution.
                \item In line \ref{findSubstates}, function \texttt{findSubstates} searches and returns
                    paths to pushable boxes (substates).
                \item In line \ref{Update}, function \texttt{pushBox} creates a new state by 
                    pushing the adjacent box in the direction substate specified.
                \item In line \ref{seq}, combining the action sequences of \texttt{currState} 
                    and \texttt{substate} that achieved the goal gives the solution.
            \end{itemize}

        \subsection{Finding Substates}
            \begin{algorithmic}[1]
                \State $Queue \gets \{x, y, ""\}$ \label{queue}
                \State $visited \gets \{x, y\}$
                \While{$Queue$ is not empty}
                    \State $x, y, seq \gets Queue.pop()$
                    \For{$move \in$ 4 directions} \label{move}
                        \State $nx \gets x + move.x, ny \gets y + move.y$
                        \State $tile \gets gameMap[nx][ny]$
                        \If{$tile$ is walkable \textbf{AND} $\{nx, ny\} \notin visited$}
                            \State $visited \gets \{nx, ny\}$
                            \State $Queue \gets \{nx, ny, seq + move.key\}$
                        \ElsIf{$tile$ is a box \textbf{AND} box is pushable}
                            \State $substates \gets \{nx, ny\}$
                        \EndIf
                    \EndFor
                \EndWhile
                \State \Return $substates$
            \end{algorithmic}

            \begin{itemize}
                \item This is a typical BFS Search algorithm to find all paths to pushable boxes.
                \item In line \ref{queue}, \texttt{Queue} contains \texttt{substate}s, 
                    which consists of the player's position and the action sequence to arrive there.
                \item In line \ref{move}, \texttt{move} includes the direction and the action key.
            \end{itemize}

        \subsection{Heuristics}

            \subsubsection{Description}
                A heuristic is an approximating index of how near a \texttt{state} is to solution.
                Given the index, we can pick \texttt{state}s with higher possiblity to be solved, 
                in an attempt to find the solution faster.

            \subsubsection{Formula}
                \begin{equation*}
                    \sum_{bp \in boxPos} min_{tp \in targetPos} (|tp_x - bp_x| + |tp_y - bp_y|)
                \end{equation*}

                \begin{itemize}
                    \item $bp$ is the position of an unsolved box.
                    \item $tp$ is the position of a target position.
                    \item The index is the sum of the minimum of all Hamiltonian distances 
                        from targets to unsolved boxes.
                \end{itemize}

        \subsection{Deadlock Detection}
            \begin{algorithmic}[1]
                \Function{isDeadlock}{gameMap, visited, x, y}
                    \State $visited \gets \{x, y\}$
                    \For{$move \in$ 4 directions}
                        \State $nx \gets x + move.x, ny \gets y + move.y$
                        \If{$gameMap[nx][ny]$ is a box \textbf{AND} $\notin visited$}
                            \State $blocked[move.key] \gets isDeadlock(gameMap, visited, nx, ny)$ \label{recursion}
                        \Else
                            \State $blocked[move.key] \gets (gameMap[nx][ny] \text{ is wall})$ \label{base}
                        \EndIf
                    \EndFor

                    \State \Return $(blocked[W] \textbf{ OR } blocked[S]) \textbf{ AND } 
                        (blocked[A] \textbf{ OR } blocked[D])$ \label{blocked}
                \EndFunction
            \end{algorithmic}

            \begin{itemize}
                \item This is a recursive algorithm to check for deadlock. 
                    The idea is for a tile, it is movable only if its depending neighbors are movable.
                \item In line \ref{recursion}, we explore the box tiles, which the current tile depends on to move.
                \item In line \ref{base}, a tile is unmovable only if it is a wall block.
                \item In line \ref{blocked}, a tile is in deadlock if two adjacent neighbors (e.g. WA, SD) are blocked.
            \end{itemize}
        
        \subsection{Parallelization}
            Since the algorithm includes multiple explorations of its neighbors, 
            almost all \texttt{for} loops are parellelized. 
            However, recording data structures (i.e. \texttt{visited}) and queues is not parallelized, 
            so they are put in critical sections.

    \section{Difficulties and Solution}
        \paragraph{Nature of the problem}
            Sokoban solving is a NP-Hard PSPACE-complete problem. Naturally, it is challenging.
        \paragraph{Designing Data Structures}
            A data structure for storing state information should be readable yet fast in execution.
            In my experience, making it object-oriented takes great toll on the performance.
        \paragraph{Parallelization}
            Parallelization requires careful design of algorithms. \\
            It is also exceptionally difficult to debug due to all threads run in parallel, 
            making debug message unreadable.
        \paragraph{Depth of Substates}
            BFS reaches all reachable boxes no matter how many steps are taken.
            This creates many redundant substates that are too deep, thus cutting the performance.
            However, limiting the depth may make searching for long answers 
            taking more iterations or even impossible.
        \paragraph{Designing Heuristic}
            Heuristics need to be accurate while not being an overhead of computation.
        \paragraph{Designing Deadlock Detection}
            When designing deadlock detection, I tend to consider way too many cases, 
            making the algorithm unable to find a solution.

    \section{Pthread vs. OpenMP}
        \subsection{Comparison}
            \begin{center}
                \begin{tabular}{|c|c|c|}
                    \hline
                    Library & Pthread & OpenMP \\
                    \hline
                    Level & low-level, fine-grained control & high-level construct \\
                    \hline
                    Scalability & Limited & Highly scalable \\
                    \hline
                    Flexiblity & Decent & Limited \\
                    \hline
                \end{tabular}
            \end{center}

        \subsection{Why I Chose OpenMP}
            With zero-to-none knowledge of low-level thread management, it is impossible to learn and 
            have decent command of Pthread in 2 weeks, since it takes time to 
            design and implement the Sokoban algorithm. On the other hand, OpenMP is friendlier in terms of 
            usage and I happened to finish the tutorial on the weekends after the instructor introduced OpenMP.
            Thus, OpenMp is the obvious choice.
\end{document}
